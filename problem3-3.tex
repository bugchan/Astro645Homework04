\section{Off the grid... (contd)}

Another method for interpolating on an irregular grid is Delaunay triangulation. The triangles need to have the following constraints (given in class 10):
\begin{itemize}
    \item The areas of the complete set of triangles fully span the space of interest,
    \item the circle circumscribing the vertices of\\
    each triangle does not enclose any other support points, and
    \item the interior angles of the triangles are maximal.
\end{itemize}

In figure \ref{fig:TriPartial}, the plot on the left is the triangulation of the full image that is being interpolated. To confirm that the triangulation obeys the constraint, a zoom of the triangulation is shown on the plot on the right. In this zoom we can confirm that the constraints are being followed. 

\begin{figure*}
    \centering
    \includegraphics{CodeAndFigures/hw04p3TriPartial.png}
    \caption{Complete Delaunay triangulation and a zoom of the triangulation showing the middle section.}
    \label{fig:TriPartial}
\end{figure*}

Once the triangulation is done, we can interpolate within the triangles made. This interpolation is a linear barycentric interpolation. This means it uses a coordinate transformation and some weights to get the value at the desired point.\footnote{More info on Ch.21.3 of NR}

Important to note this interpolation is not high precision because:
\begin{itemize}
    \item It is continuous within the triangles but not from triangle to triangle.
    \item On the edge of the triangle, it only uses the two vertices at the end of the edge to interpolate.
\end{itemize}

Using the image of the previous problem and removing 75\% of the original pixels, we can use this triangulation to interpolate the image. Figure \ref{fig:OrigRecons} shows the original image, the base image used for interpolation and the reconstructed image after using the Delaunay interpolation. Using the same method to calculate the error, it is estimated to be 19.91 which is higher in comparison to the RBF interpolation done in problem 2.

\begin{figure*}
    \centering
    \includegraphics{CodeAndFigures/hw04p3OrigPlusReconstructed.png}
    \caption{In the left, is the original image. In the middle, is the image with 75\% of the pixels removed. In the right, the interpolated image.}
    \label{fig:OrigRecons}
\end{figure*}