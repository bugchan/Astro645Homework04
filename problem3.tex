\section{Orbits in an axisymmetric potential}
Considering an axisymmetric potential for a simple model of a galaxian disk first proposed by Toomre (1964, ApJ):
\begin{equation}
    U(r) = -(1+r^2)^{-1/2},
\end{equation}

the equation of motion is given by
\begin{equation}
    F=-\frac{dU(r)}{dr}=-\frac{r}{(1+r^2)^{3/2}}.
\end{equation}

This is a differential equation that can be solved using leapfrog integration.

\begin{figure*}[h!]
    \centering
    \includegraphics{CodeAndFigures/ToomrePotentialOrbits.pdf}
    \caption{Rosette Orbits using initial conditions of Left: $x=.3$, $y=0$, $v_x=.3$ and $v_y=.4$ Middle: $x=0$, $y=0.5$, $v_x=.6$ and $v_y=0$. Right:$x=1$, $y=0$, $v_x=0$ and $v_y=1$.}
    \label{fig:ToomreOrbits}
\end{figure*}

Using the initial values listed on table \ref{tab:ToomreOrbits}, we can calculate the orbits for each case. Figure \ref{fig:ToomreOrbits} shows that the orbits are not close but they precess. These types of orbits are called rosette orbits because of the shape or tube orbits because they have an inner and outer boundary.

\begin{table}
  \begin{center}
    \caption{Initial values used for calculating each rosette orbit using the Toomre potential.}
    \label{tab:ToomreOrbits}
    \pgfplotstabletypeset[
      multicolumn names, % allows to have multicolumn names
      col sep=comma, % the separator in our .csv file
      display columns/0/.style={
		column name=Orbit, % name of first column
		%column type={S},string type
		},  % use siunitx for formatting
      display columns/1/.style={
		column name=$x$,
		%column type={S},string type
		},
	  display columns/2/.style={
		column name=$y$,
		%column type={S},string type
		},
	  display columns/3/.style={
		column name=$v_x$,
		%column type={S},string type
		},
	 display columns/4/.style={
		column name=$v_y$,
		%column type={S},string type
		},
      every head row/.style={
		before row={\toprule}, % have a rule at top
		after row={
% 			\si{\ampere} & \si{\volt}\\ % the units seperated by &
 			\midrule} % rule under units
 			},
	  every last row/.style={after row=\bottomrule}, % rule at bottom
    ]{CodeAndFigures/ToomreOrbitsData.csv} % filename/path to file
  \end{center}
\end{table}

These systems have two conserved quantities, the total energy $E_T$ and angular momentum $L$. Using the initial conditions, one can calculate the total energy of the system,
\begin{equation}
    E_T=K + U(r) = \frac{1}{2}v^2 + U(r)
\end{equation}
where U is the potential energy and K is the kinetic energy of the initial values, and the angular momentum by
\begin{equation}
    L=r^2\Dot{\theta}
\end{equation}
where 
\begin{align}
    r^2 = x^2 + y^2\\
    \Dot{\theta} = r \times \dot{r}\\
    \dot{r}^2 = v_x^2 + v_y^2
\end{align}.

Figure \ref{fig:conservedQuants} shows the total energy and the angular momentum of these rosette orbits are being conserved through time. 

The inner and outer radius of the orbits are found by finding the roots of
\begin{equation}
    E_T - U(r) - \frac{L^2}{r^2} = 0
\end{equation}


%%%%%%%%%%%%%%%%%%%%%%%%%%%%%%%%%%%%%%%%%%%%%%%%%%%%%%%%%%%%%%%
%=========================SUBSECTION===========================
%%%%%%%%%%%%%%%%%%%%%%%%%%%%%%%%%%%%%%%%%%%%%%%%%%%%%%%%%%%%%%%
\subsection{}
% (b) Compute the energy and angular momentum of your trial orbits
% (e.g. using the initial conditions). Compute the inner and outer
% radii of the tube (the turning points) using the conserved quantities
% and check these values against your direct orbit integration.



\begin{figure*}
    \centering
    \includegraphics{CodeAndFigures/EnergyMomentumPlot.pdf}
    \caption{Left: Energy and Right: Angular Momentum of the rosette orbits in function with time.}
    \label{fig:conservedQuants}
\end{figure*}
