\section{Irregular orbits: Part 1}
Most often, one does not know the fraction of irregular orbits for particular density model. A very useful technique for both determining whether an orbit is regular or irregular and classifying the type of orbit
is the “surface of section” method. This is described in detail in B\&T section 3.2.2.

%%%%%%%%%%%%%%%%%%%%%%%%%%%%%%%%%%%%%%%%%%%%%%%%%%%%%%%%%%%%%%%
%=========================SUBSECTION===========================
%%%%%%%%%%%%%%%%%%%%%%%%%%%%%%%%%%%%%%%%%%%%%%%%%%%%%%%%%%%%%%%
\subsection{}
Adapt your ODE program to find the $x-\Dot{x}$ surface of section that
is, have your program plot with points the the values of $x$ and $\Dot{x}$
when $y = 0$ and $\Dot{y}> 0$.




%%%%%%%%%%%%%%%%%%%%%%%%%%%%%%%%%%%%%%%%%%%%%%%%%%%%%%%%%%%%%%%
%=========================SUBSECTION===========================
%%%%%%%%%%%%%%%%%%%%%%%%%%%%%%%%%%%%%%%%%%%%%%%%%%%%%%%%%%%%%%%
\subsection{}
Use your ODE program to find a tube orbit and a box orbit in the
potential given by B\&T equation 3-103 with $R_c = 0.14$, $q = 0.9$
and $v_c = 1$. (See Figure 3-8 for inspiration).




%%%%%%%%%%%%%%%%%%%%%%%%%%%%%%%%%%%%%%%%%%%%%%%%%%%%%%%%%%%%%%%
%=========================SUBSECTION===========================
%%%%%%%%%%%%%%%%%%%%%%%%%%%%%%%%%%%%%%%%%%%%%%%%%%%%%%%%%%%%%%%
\subsection{}
Now, plot the surface of section for these orbits. Note/discuss/explain
the differences. [We will explore the surface of sections for irregular orbits in a later PS.]

\clearpage