\section{Symplectic integration.}

% Generalize your ODE program for integrating and plotting the equations of motion using the symplectic leapfrog integrator. I did not have
% time to discuss this algorithm in class today, but I’m attaching a brief
% introduction as an appendix to get you started. I will discuss this at the
% beginning of class next Thursday.
A symplectic integrator is an integrator that conserve phase-space volume and Poincar\' e invariants \cite{BinneyT_}

Explain Leapfrog 





%%%%%%%%%%%%%%%%%%%%%%%%%%%%%%%%%%%%%%%%%%%%%%%%%%%%%%%%%%%%%%%
%=========================SUBSECTION===========================
%%%%%%%%%%%%%%%%%%%%%%%%%%%%%%%%%%%%%%%%%%%%%%%%%%%%%%%%%%%%%%%
\subsection{}
% (a) As a test, use this integrate the pendulum equations of motion in
% all regimes:
% i. rotation,
% ii. libration,
% iii. near the unstable equilibrium
We can test our leapfrog code by integrating the equation of motion of the pendulum. 

The equation of motion of the pendulum is given by
\begin{equation}
    \Ddot{\theta}=-\sin\theta.
\end{equation}

We can determine 3 scenarios:
\begin{itemize}
    \item Rotation
    Rotation is when the total energy given by
    \begin{equation}
    E=\frac{1}{2}\Dot{\theta}^2+\left(1+\cos\theta\right)
    \end{equation}
    is bigger than the critical energy. The critical energy is 
    
    \item Libration
    \item Near Unstable Equilibrium
\end{itemize}

\begin{table}[]
    \centering
\begin{tabular}{lrr}
\toprule
{} &  $\theta$ &  $\dot{\theta}$ \\
\midrule
Rotation    &     -4.08 &             1.30 \\
Libration   &      0.50 &             0.00 \\
Near Uns Eq &      3.13 &             0.00 \\
\bottomrule
\end{tabular}

    \caption{Caption}
    \label{tab:PendulumIV}
\end{table}



%%%%%%%%%%%%%%%%%%%%%%%%%%%%%%%%%%%%%%%%%%%%%%%%%%%%%%%%%%%%%%%
%=========================SUBSECTION===========================
%%%%%%%%%%%%%%%%%%%%%%%%%%%%%%%%%%%%%%%%%%%%%%%%%%%%%%%%%%%%%%%
\subsection{}
(b) Demonstrate that the leapfrog integrator is second-order accurate
in the sense that errors in $\mathbf{q}$ and $\mathbf{p}$ after a timestep $h$ are $O(h^3)$.



%%%%%%%%%%%%%%%%%%%%%%%%%%%%%%%%%%%%%%%%%%%%%%%%%%%%%%%%%%%%%%%
%=========================SUBSECTION===========================
%%%%%%%%%%%%%%%%%%%%%%%%%%%%%%%%%%%%%%%%%%%%%%%%%%%%%%%%%%%%%%%
\subsection{}
(c) Compute the energy conservation for the trajectories in each regime
over about 10,000 orbital periods. Compare the energy conservation for RK4 and leapfrog.