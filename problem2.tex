\section{Symplectic integration.}

Generalize your ODE program for integrating and plotting the equations of motion using the symplectic leapfrog integrator. I did not have
time to discuss this algorithm in class today, but I’m attaching a brief
introduction as an appendix to get you started. I will discuss this at the
beginning of class next Thursday.
%%%%%%%%%%%%%%%%%%%%%%%%%%%%%%%%%%%%%%%%%%%%%%%%%%%%%%%%%%%%%%%
%=========================SUBSECTION===========================
%%%%%%%%%%%%%%%%%%%%%%%%%%%%%%%%%%%%%%%%%%%%%%%%%%%%%%%%%%%%%%%
\subsection{}
(a) As a test, use this integrate the pendulum equations of motion in
all regimes:
i. rotation,
ii. libration,
iii. near the unstable equilibrium




%%%%%%%%%%%%%%%%%%%%%%%%%%%%%%%%%%%%%%%%%%%%%%%%%%%%%%%%%%%%%%%
%=========================SUBSECTION===========================
%%%%%%%%%%%%%%%%%%%%%%%%%%%%%%%%%%%%%%%%%%%%%%%%%%%%%%%%%%%%%%%
\subsection{}
(b) Demonstrate that the leapfrog integrator is second-order accurate
in the sense that errors in $\mathbf{q}$ and $\mathbf{p}$ after a timestep $h$ are $O(h^3)$.



%%%%%%%%%%%%%%%%%%%%%%%%%%%%%%%%%%%%%%%%%%%%%%%%%%%%%%%%%%%%%%%
%=========================SUBSECTION===========================
%%%%%%%%%%%%%%%%%%%%%%%%%%%%%%%%%%%%%%%%%%%%%%%%%%%%%%%%%%%%%%%
\subsection{}
(c) Compute the energy conservation for the trajectories in each regime
over about 10,000 orbital periods. Compare the energy conservation for RK4 and leapfrog.