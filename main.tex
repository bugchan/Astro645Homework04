\documentclass[12pt,letterpaper]{article}
\usepackage{fullpage}
\usepackage[top=2cm, bottom=4cm, left=2cm, right=2cm]{geometry}
\usepackage{amsmath,amsthm,amsfonts,amssymb,amscd}
\usepackage{lastpage}
\usepackage{enumerate}
\usepackage{fancyhdr}
\usepackage{mathrsfs}
\usepackage{xcolor}
\usepackage{graphicx}
\usepackage{listings}
\usepackage{hyperref}
\usepackage{siunitx}
\usepackage{appendix}
\usepackage{caption}
\usepackage{subcaption}
\usepackage{multicol}
\usepackage{wrapfig}
\usepackage{esint}
\usepackage[utf8]{inputenc}
\usepackage[inline]{enumitem} %allows inline itemize or enumerate


\hypersetup{%
  colorlinks=true,
  linkcolor=blue,
  linkbordercolor={0 0 1}
}
 
\renewcommand\lstlistingname{Algorithm}
\renewcommand\lstlistlistingname{Algorithms}
\def\lstlistingautorefname{Alg.}

\lstdefinestyle{Python}{
    language        = Python,
    frame           = lines, 
    basicstyle      = \tiny,
    keywordstyle    = \color{blue},
    stringstyle     = \color{green},
    commentstyle    = \color{orange}\ttfamily
}

\setlength{\parindent}{0.0in}
\setlength{\parskip}{0.1in}
\setlength{\tabcolsep}{6pt}

%Astronomical units
\DeclareSIUnit \parsec  {pc}
\DeclareSIUnit \year    {yr}

% Edit these as appropriate
% \newcommand\course{ASTRO 732}
% \newcommand\coursename{Computational Methods for Astrophysics}
\newcommand\course{ASTRO 645}
\newcommand\coursename{Astrophysical Dynamics}
\newcommand\hwnumber{4}
\newcommand\myname{Sandra Bustamante}  
%allows to number the last equation of align.
\newcommand\numberthis{\addtocounter{equation}{1}\tag{\theequation}}

\pagestyle{fancyplain}
\headheight 35pt
\lhead{\course \\ Homework \hwnumber}
\chead{\textbf{\Large Potential Theory}}
\rhead{\myname \\ Due October 29, 2019}
\lfoot{}
\cfoot{\coursename}
\rfoot{\small\thepage}
\headsep 1.5em

%redefines subsections to be letters instead of numbers so 1.a normal way is 1.1
% \arabic (1, 2, 3, ...)
% \alph (a, b, c, ...)
% \Alph (A, B, C, ...)
% \roman (i, ii, iii, ...)
% \Roman (I, II, III, ...)
\renewcommand\thesubsection{\thesection.\alph{subsection}}

%%%%%%%%%%%%%%%%%%%%%%%%%%%%%%%%%%%%%%%%%%%%%%%%%%%%%%%%%%%%%%%
%=====================BEGIN DOCUMENT===========================
%%%%%%%%%%%%%%%%%%%%%%%%%%%%%%%%%%%%%%%%%%%%%%%%%%%%%%%%%%%%%%%

\begin{document}
\twocolumn
\lstset{style=Python}

% textwidth is \the\textwidth \\
% column width is \the\columnwidth \\
% textheight is \the\textheight

\section{The 1-D Warmup}

Interpolation is the process of finding the values within the known points, also called support points. We can do a local interpolation, which is just using nearby support points or a global interpolation using all support points. 

Linear interpolation uses a straight line between the support points.

Polynomial interpolation, as its name, uses a polynomial function where the coefficients are linear. The coefficient can be found by constructing the following matrices and using LU decomposition or SVD:
\begin{equation}
    \begin{bmatrix}
        1.0 &   x_0 &   x_0^2 & \dots &x_0^n\\
        1.0 &   x_0 &   x_0^2 & \dots &x_0^n\\
        \vdots& \vdots& \vdots&         &\vdots\\
        1.0 &   x_n &   x_n^2 & \dots &x_n^n\\
    \end{bmatrix}
    \cdot
    \begin{bmatrix}
        c_0   \\
        c_1   \\
        \vdots   \\
        c_n   \\
    \end{bmatrix}
    =
    \begin{bmatrix}
        y_0  \\
        y_1 \\
        \vdots \\
        y_n \\
    \end{bmatrix}
\end{equation}

Another method to do a polynomial interpolation is by using Neville's Algorithm. This method first finds the polynomial of degree 0 that fits the support points then uses that information to find the polynomial of a higher degree. It continues recursively until it find the polynomial of the degree desired. In this way it creates a tableau like to find the value of the desired point. 

The polynomial interpolation error can be calculated for the upper limits as
\begin{equation}
    f(x)-P_n(x) = \frac{1}{(n+1)!}f^{n+1}(\Psi)\prod^n_{i=0}(x-x_i)
    \label{eq:polyError}
\end{equation}

where n is the order of the polynomial function and $x_i$ are the support points. 

Cubic Spline Interpolation uses the constraints that the interpolating functions needs to be \\
smooth on both sides of a point and the second derivative is continuous. Optionally you can add another constraint that specifies the behavior of the endpoints. For this case, the code uses the natural boundary which means that the spline is a straight line at these points.
\begin{figure}
    \centering
    \includegraphics{CodeAndFigures/hw04p1Plot.png}
    \caption{Plot that shows four different interpolations within the same support points.}
    \label{fig:p1interp}
\end{figure}

Figure \ref{fig:p1interp} shows the support points given for this problem and the linear, polynomial of order 6, polynomial using Neville's algorithm and cubic spline interpolation. Figure \ref{fig:p1interpError} shows the fractional error for each of the interpolations. The fractional error was calculated as follows
\begin{equation}
    \mathrm{error}_{frac}=1-\frac{\mathrm{Interp}}{y(x)}
\end{equation}

 where $y(x)$ is the real values of the function given by
\begin{equation}
    y(x) = \frac{\sin{x}}{x}
\end{equation}

\begin{figure}
    \centering
    \includegraphics{CodeAndFigures/hw04p1Error.png}
    \caption{The fractional error of four different interpolations: Linear, Polynomial using SVD, Polynomial using Neville's and Cubic Spline.}
    \label{fig:p1interpError}
\end{figure}

As expected, the linear error consistently gives higher errors followed by the polynomial interpolation using SVD, then by the cubic spline and finally the polynomial interpolation using Neville's algorithm gives the lowest error out of the four. Additionally, we can see that at the support points the error is minimum showing that the interpolation is very close to the known values as expected of a good interpolation.

\begin{figure}
    \centering
    \includegraphics{CodeAndFigures/hw04p1ErrorPolynomial.png}
    \caption{Comparison of the fractional error of the polynomial interpolation using SVD and Neville's algortihm and the upper bound polynomial error calculated from equation \ref{eq:polyError}.}
    \label{fig:p1PolyError}
\end{figure}

Figure \ref{fig:p1PolyError} shows a comparison of the fractional error of the polynomial obtained from SVD and Neville's algorithm and the upper bound error calculated using equation \ref{eq:polyError} for the polynomial interpolation.

We can see that the upper bound is overall lower than the fractional error of the interpolation using SVD which is not what it is expected. The upper bound error is an estimate of the maximum error we can expect to have and I expected to be over the fractional error of polynomial interpolation using SVD. After revising my code, I couldn't find what I did wrong to get this outcome. I suspect that since we are using a polynomial of order 6 and have 9 support points, my system is over-determined. In this case, my coefficients are the least square solution of the matrix and because of this is not exact at the support points.

In contrast, the upper bound error is higher than the fractional error from the polynomial interpolation using Neville's algorithm. This behavior is the expected. 

Figure \ref{fig:p1PolyError6} shows the same information as \ref{fig:p1PolyError}. In this we are repeating the same approach of calculating the coefficients using SVD but selecting only a subset of 6 consecutive support points. In this case, the fractional error is mostly lower than the polynomial error for the selected range of support points. Outside of the range the error gets higher. This is expected since we are interpolating and have no information outside of the support points used.

\begin{figure}
    \centering
    \includegraphics{CodeAndFigures/hw04p1ErrorPolynomial6.png}
    \caption{Example of how selecting only 7 values of the support points for the polynomial interpolation changes the fractional error to be lower than the upper bound error for those points.}
    \label{fig:p1PolyError6}
\end{figure}
\clearpage

% \section{Off the grid...}

A form of interpolation on an irregular grid is Radial Basis Interpolation (RBF). It uses the idea that data points are only influence by \\
nearby points by using functional forms that are a function of the distance from each point. 

Some radial basis functions are:
\begin{align}
\mathrm{Multiquadric:}\; &\phi(r)= (r^2 + r_0^2)^{1/2} \\
\mathrm{Inverse:}\;&\phi(r)= (r^2 + r_0^2)^{-1/2} \\
\mathrm{Thin-plate:}\;&\phi(r)= r^2 \log\left(\frac{r}{r_0}\right) \\
\mathrm{Gaussian:}\;&\phi(r) = \exp\left(-\frac{1}{2}\frac{r^2}{r^2_0}\right)\\
\mathrm{Cubic:}\; &\phi(r)=r^3
\end{align}

where $r_0$ is a free parameter. To choose this parameter it is recommended to experiment with several values. 

In this exercise, given an image showed in Figure \ref{fig:p2Orig}, here after refer as the original image, we needed to removed 25\%, 50\% and 75\% of pixels. To achieve this, I first determined the number of pixels needed to removed for the ratio of pixels removed and total pixels correspond to the percentages given. Then I created an array of uniformly random coordinates of that length. Since some of the random coordinate repeated, the code starts to remove pixels until it reaches the desired percentage.

\begin{figure}
    \centering
    \includegraphics{CodeAndFigures/hw04p2Original.png}
    \caption{Original image}
    \label{fig:p2Orig}
\end{figure}

After removing the desired percentage of pixels, I created the base images for interpolating with different radial functions. The 3 radial basis functions selected were quadratic, inverse and thin plate. 

After this, the code determines the best r0 for each case. The range of r0 to be evaluated was determined by a manual process of creating a coarse grid of r0 values ranging from .1 to 15 and plotting them. After viewing the plots, I determined that the best fit was within .1 and 1.5 for all of my functions. Then, this range is used to find the best r0 in a finer grid.

To determine the best fit, it compared the RMS of the square difference between the reconstructed image $I_r$ and original image $I_0$ by:
\begin{equation}
    \mathrm{error}_\mathrm{RMS}=\sqrt{\frac{\sum(I_0 - I_r)^2}{N}}
\end{equation}
where $N$ is the total number of pixels in the original image. 
Figure\ref{fig:p2Error} shows three plots for each base image. Each plot shows how the RMS changes depending of the value of r0. Note that for the thin plate case, the RMS is constant since the implementation of this radial function on scipy does not depend of r0.  
\begin{figure*}
    \centering
    \includegraphics{CodeAndFigures/hw04p2ErrorPlot.png}
    \caption{Plot that showed the RMS values  }
    \label{fig:p2Error}
\end{figure*}

After determining the best r0, it selects the corresponding reconstructed image and plots it. \\
Figure \ref{fig:p2InterAll} shows the base image and the reconstructed image with the lowest RMS for each case. From this comparison we can note that the best interpolation for all base images is achieved by using the multiquadric function. 

\begin{figure*}
    \centering
    \includegraphics{CodeAndFigures/hw04p2InterpolatedAll.png}
    \caption{Base Image and the best reconstructed image with the lowest RMS. It shows the corresponding error RMS achieved for each case and the optimal r0 for the multiquadric and inverse radial functions.}
    \label{fig:p2InterAll}
\end{figure*}
% \clearpage

% \section{Off the grid... (contd)}

Another method for interpolating on an irregular grid is Delaunay triangulation. The triangles need to have the following constraints (given in class 10):
\begin{itemize}
    \item The areas of the complete set of triangles fully span the space of interest,
    \item the circle circumscribing the vertices of\\
    each triangle does not enclose any other support points, and
    \item the interior angles of the triangles are maximal.
\end{itemize}

In figure \ref{fig:TriPartial}, the plot on the left is the triangulation of the full image that is being interpolated. To confirm that the triangulation obeys the constraint, a zoom of the triangulation is shown on the plot on the right. In this zoom we can confirm that the constraints are being followed. 

\begin{figure*}
    \centering
    \includegraphics{CodeAndFigures/hw04p3TriPartial.png}
    \caption{Complete Delaunay triangulation and a zoom of the triangulation showing the middle section.}
    \label{fig:TriPartial}
\end{figure*}

Once the triangulation is done, we can interpolate within the triangles made. This interpolation is a linear barycentric interpolation. This means it uses a coordinate transformation and some weights to get the value at the desired point.\footnote{More info on Ch.21.3 of NR}

Important to note this interpolation is not high precision because:
\begin{itemize}
    \item It is continuous within the triangles but not from triangle to triangle.
    \item On the edge of the triangle, it only uses the two vertices at the end of the edge to interpolate.
\end{itemize}

Using the image of the previous problem and removing 75\% of the original pixels, we can use this triangulation to interpolate the image. Figure \ref{fig:OrigRecons} shows the original image, the base image used for interpolation and the reconstructed image after using the Delaunay interpolation. Using the same method to calculate the error, it is estimated to be 19.91 which is higher in comparison to the RBF interpolation done in problem 2.

\begin{figure*}
    \centering
    \includegraphics{CodeAndFigures/hw04p3OrigPlusReconstructed.png}
    \caption{In the left, is the original image. In the middle, is the image with 75\% of the pixels removed. In the right, the interpolated image.}
    \label{fig:OrigRecons}
\end{figure*}
% \newpage

%%%%%%%%%%%%%%%%%%%%%%%%%%%%%%%%%%%%%%%%%%%%%%%%%%%%%%%%%%%%%%%
%=====================Add Bibliography=========================
%%%%%%%%%%%%%%%%%%%%%%%%%%%%%%%%%%%%%%%%%%%%%%%%%%%%%%%%%%%%%%%

% \bibliographystyle{plain} % We choose the "plain" reference style
% %plain style sorts the reference list by alphabetical order of the first author’s last name.
% \bibliography{references} % Entries are in the "refs.bib" file

% % %\include{name} %name without extension. insert in new page

% \clearpage


%%%%%%%%%%%%%%%%%%%%%%%%%%%%%%%%%%%%%%%%%%%%%%%%%%%%%%%%%%%%%%%
%=====================Appendix=================================
%%%%%%%%%%%%%%%%%%%%%%%%%%%%%%%%%%%%%%%%%%%%%%%%%%%%%%%%%%%%%%%
% \appendix

% \lstset{caption={Astro732\_HW04P1.py}, style=Python}
% \section[]{Python code Problem 1} 
% \label{sec:CodeHw04p1}
% \lstset{label={Astro732Hw04P1.py}}
% \lstinputlisting[language=Python]{CodeAndFigures/Astro732_HW04P1.py}

% \clearpage

% \lstset{caption={Astro732\_Hw04P2.py}, style=Python}
% \section[]{Python code Problem 2} 
% \label{sec:CodeHw04p2}
% \lstset{label={Astro732HW04P2BestR0.py}}
% \lstinputlisting[language=Python]{CodeAndFigures/Astro732_HW04P2_BestR0.py}

% \clearpage

% \lstset{caption={Astro732\_Hw04p3.py}, style=Python}
% \section[]{Python code Problem 3} 
% \label{sec:CodeHw03p3}
% \lstset{label={Astro732Hw04p3.py}}
% \lstinputlisting[language=Python]{CodeAndFigures/Astro732_HW04P3.py}


\end{document}