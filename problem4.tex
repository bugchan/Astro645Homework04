\section{Non-axisymmetric orbits}

Finally, a more complicated non-axisymmetric case—two fixed point
masses—which has the potential:
\begin{equation}
    U(x,y) = -[(x-a)^2 +y^2]^{-1/2} -[(x+a)^2 +y^2]^{-1/2}.
\end{equation}

Although far from obvious, this potential can be separated in confocal
elliptical coordinates and put in a Staeckel form (as described in B\&T
and therefore has regular orbits. If the two point masses were in orbit
around each other, the orbits would not be regular. In other words, the
real three body problem has chaotic trajectories!
%%%%%%%%%%%%%%%%%%%%%%%%%%%%%%%%%%%%%%%%%%%%%%%%%%%%%%%%%%%%%%%
%=========================SUBSECTION===========================
%%%%%%%%%%%%%%%%%%%%%%%%%%%%%%%%%%%%%%%%%%%%%%%%%%%%%%%%%%%%%%%
\subsection{}
% (a) Again, pick some initial conditions and integrate the orbits. Just
% to be definite a = 1=2 (although the problem scales with a so you
% can choose value you wanted and get the same results).

The equations of motion for this potential are
\begin{align*}
    F&=-\nabla U(r) =-(\frac{\partial U(r)}{\partial x}+\frac{\partial U(r)}{\partial y}\\
    \Ddot{x}&=-\left[\frac{x-a}{(x-a)^2+y^2}+\frac{x+a}{(x+a)^2+y^2}\right] \\
    \Ddot{y}&=- \left[\frac{y}{(x-a)^2+y^2}+\frac{y}{(x+a)^2+y^2}\right]
\end{align*}

This can be evaluated using the initial conditions listed in table \ref{tab:NonAxisSymetric} and $a= \frac{1}{2}$.


\begin{table}[]
    \centering
\input{CodeAndFigures/NonAxisSymetricData.tex}
    \caption{Caption}
    \label{tab:ToomreOrbitIV}
\end{table}

%%%%%%%%%%%%%%%%%%%%%%%%%%%%%%%%%%%%%%%%%%%%%%%%%%%%%%%%%%%%%%%
%=========================SUBSECTION===========================
%%%%%%%%%%%%%%%%%%%%%%%%%%%%%%%%%%%%%%%%%%%%%%%%%%%%%%%%%%%%%%%
\subsection{}
(b) There are two different kinds of orbits in this potential: tube orbits
and box orbits. Box orbits come arbitrarily close to the center of
force filling a closed area (similar to Lissajous figures). In fact,
there are two cases of box orbits here: those which come close to
one and those which come close to both centers of force. Find an
example of each case.
%%%%%%%%%%%%%%%%%%%%%%%%%%%%%%%%%%%%%%%%%%%%%%%%%%%%%%%%%%%%%%%
%=========================SUBSECTION===========================
%%%%%%%%%%%%%%%%%%%%%%%%%%%%%%%%%%%%%%%%%%%%%%%%%%%%%%%%%%%%%%%
\subsection{}
(c) Extra credit Investigate the relationship between conserved quantities and the envelope of the orbits. Report your findings.
Note: because the force is generated by two point masses, box orbits
will come arbitrarily close to one or both of the force centers. This
makes the solution numerically tricky. Consider using small error tolerances and/or small time steps . . .